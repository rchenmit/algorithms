
\documentclass{article}

\usepackage{graphicx}




\usepackage{subfigure}

\usepackage{float}


\textheight=9.5in
\textwidth=6.5in
\topmargin=-.8in
\headsep=0pt
\oddsidemargin=0truecm
\evensidemargin=0truecm
\footskip=20pt
%\footheight=20pt
%\pretolerance=1600
%\tolerance=1600
%\hbadness=1600





\newtheorem{theorem}{Theorem}[section]
\newtheorem{corollary}[theorem]{Corollary}
\newtheorem{lemma}[theorem]{Lemma}
\newtheorem{proposition}[theorem]{Proposition}
\newtheorem{conjecture}[theorem]{Conjecture}
\newtheorem{problem}[theorem]{Problem}

\begin{document}

\title{
}

%\author{Assigned 1-13-16 , Due 1-19-16 (in class)
%}


\date{}

%%%%%%%%%%%%%%%%%%%%%%%%%%%%%%%%%%%%%%%%%%%%%%%%%%%%%%%%
% Author's definitions

\newcommand{\DEF}[1]{{\em #1\/}}

\newcommand\chic{\chi_c}
\newcommand\C{\hbox{${\cal C}$}}
\newcommand{\RR}{\mbox{$\mathbb R$}}
\newcommand{\NN}{\mbox{$\mathbb N$}}
\newcommand{\ZZ}{\mbox{$\mathbb Z$}}
\newcommand{\eopf}{\raisebox{0.8ex}{\framebox{}}}
\newcommand{\dist}{\hbox{\rm d}}
\renewcommand\a{\alpha}
\renewcommand\b{\beta}
\renewcommand\c{\gamma}
\renewcommand\d{\delta}
\newcommand\D{\Delta}
\newcommand{\directedchi}{\mbox{$\vec{\chi}$}}
\newcommand{\directedE}{\mbox{$\vec{E}$}}
\newcommand{\directedG}{\mbox{$\vec{G}$}}
\newcommand{\directedK}{\mbox{$\vec{K}$}}

\newenvironment{proof}%
{\noindent{\bf Proof.}\ }%
{\hfill\eopf\par\bigskip}%

%%%%%%%%%%%%%%%%%%%%%%%%%%%%%%%%%%%%%%%%%%%%%%%%%%%%%%%%

%\maketitle


\noindent{\bf  Last Name:} $.............................$
\noindent{\bf First Name:} $..............................$
\noindent{\bf Email:}          $..............................$

\noindent
\noindent{ CS 6505, Fall 2017, Homework 3, 9-20-17 Due 9-25-17 by 6pm Klaus 2138 $~~$Page 1/4 }\\

\noindent
{\bf Problem 1: Subset Sum, Dynamic Programming (25 points)}\\
You are given $n$ items $\{ 1 , 2 , \ldots n \}$, where each item has a given positive weight $w_i$, $1\leq i \leq n$. 
You are also given an upper bound $W$. 
You would line to select a subset $S$ of the items so that $\sum_{i\in S}w_i \leq W$, 
and, subject to this restriction,  $\sum_{i\in S}w_i \leq W$ is as large as possible. 
Give an $O(nW)$ algorithm, justify correctness and running time.\\

\noindent
{\bf Answer:}\\

\pagebreak
\noindent{\bf  Last Name:} $.............................$
\noindent{\bf First Name:} $..............................$
\noindent{\bf Email:}          $..............................$

\noindent
\noindent{ CS 6505, Fall 2017, Homework 3, 9-20-17 Due 9-25-17 by 6pm Klaus 2138 $~~$Page 2/4 }\\


\noindent{\bf Problem 2: Balanced Tree, Dynamic Programming (25 points)}.\\
Let $T(V,E)$ be a directed acyclic graph with $V=\{ v_1 , \ldots v_n \}$.\\
Suppose that $T$ is given in topologically sorted order, that is, 
if $v_i$ is an ancestor of $v_j$ then $i<j$.\\
Suppose further that  each vertex $v_i \in V$ has a given positive cost $c(v_i) > 0$. 
Define the weight of a vertex $v_i \in V$ as the sum of the costs of all vertices that can be reached from $v_i$
(equivalently belong to the subtree rooted at $v_i$) : 
$weight (v_i) = \sum_{
\begin{array}{c}
v_j \in V : \\ \mbox{$v_j$ is reachable from $v_i$} \end{array}
}  c(v_j)$\\
Say that $T$ is balanced if and only if, for every vertex $v_i \in V$,
if $v_i$ has children $u_1 , \ldots , u_k$, then 
$$
weight(u_1 ) = weight(u_2) = \ldots = weight(u_k)
$$
Give a polynomial time algorithm that decides if a directed tree with costs on its vertices is balanced. \\
Justify your answer and argue running time.\\

\noindent
{\bf Answer:}\\

\pagebreak
\noindent{\bf  Last Name:} $.............................$
\noindent{\bf First Name:} $..............................$
\noindent{\bf Email:}          $..............................$

\noindent
\noindent{ CS 6505, Fall 2017, Homework 3, 9-20-17 Due 9-25-17 by 6pm Klaus 2138 $~~$Page 3/4 }\\

\noindent{\bf Problem 3: Max Independent Set, Dynamic Programming (25 points)}\\
(a) Consider a line graph on vertices $\{ 1 , \ldots , n \}$
and edges $\{ 1 , 2 \}$, $\{ 2 , 3 \}$, ... , $\{ (n-1) , n \}$. 
Each vertex has a positive weight $w_i$, $1 \leq i \leq n$. 
Give an $O(n)$ algorithm that outputs the weight of a maximum weight independent set of the line graph. 
You may give a simple description of the algorithm, and/or pseudocode. 
You should include a short argument of correctness and running time.\\
(b) Consider a cycle graph on vertices $\{ 1 , \ldots , n \}$
and edges $\{ 1 , 2 \}$, $\{ 2 , 3 \}$, ... , $\{ (n-1) , n \}$, $\{ n , 1 \}$.
Each vertex has a positive weight $w_i$, $1 \leq i \leq n$. 
Give an $O(n)$ algorithm that outputs the weight of a maximum weight independent set of the cycle graph. 
You may give a simple description of the algorithm, and/or pseudocode. 
You should include a short argument of correctness and running time.\\

\noindent
{\bf Answer:} \\


\pagebreak
\noindent{\bf  Last Name:} $.............................$
\noindent{\bf First Name:} $..............................$
\noindent{\bf Email:}          $..............................$

\noindent
\noindent{ CS 6505, Fall 2017, Homework 3, 9-20-17 Due 9-25-17 by 6pm Klaus 2138 $~~$Page 4/4 }\\

\noindent{\bf Problem 4: Longest Path, Dynamic Programming (25 points)}\\
Let $G(V,E)$ be a directed acyclic graph, where $V=\{ v_1 , \ldots , v_n \}$.
The graph is presented in adjacency list representation, and with the property 
that $v_i \rightarrow v_j \in E$ only if $i < j$. 
Give an $O(|V|+|E|)$ algorithm that finds the length of the longest path (maximum number of edges) from $v_1$ to $v_n$. 
If there is no path from $v_1$ to $v_2$ then your algorithm should output $\infty$.\\
Give a short justification of correctness and running time. \\

\noindent
{\bf Answer:}\\

\end{document}






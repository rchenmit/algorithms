
\documentclass{article}

\usepackage{graphicx}




\usepackage{subfigure}

\usepackage{float}


\textheight=9.5in
\textwidth=6.5in
\topmargin=-.8in
\headsep=0pt
\oddsidemargin=0truecm
\evensidemargin=0truecm
\footskip=20pt
%\footheight=20pt
%\pretolerance=1600
%\tolerance=1600
%\hbadness=1600





\newtheorem{theorem}{Theorem}[section]
\newtheorem{corollary}[theorem]{Corollary}
\newtheorem{lemma}[theorem]{Lemma}
\newtheorem{proposition}[theorem]{Proposition}
\newtheorem{conjecture}[theorem]{Conjecture}
\newtheorem{problem}[theorem]{Problem}

\begin{document}

\title{
}

%\author{Assigned 1-13-16 , Due 1-19-16 (in class)
%}


\date{}

%%%%%%%%%%%%%%%%%%%%%%%%%%%%%%%%%%%%%%%%%%%%%%%%%%%%%%%%
% Author's definitions

\newcommand{\DEF}[1]{{\em #1\/}}

\newcommand\chic{\chi_c}
\newcommand\C{\hbox{${\cal C}$}}
\newcommand{\RR}{\mbox{$\mathbb R$}}
\newcommand{\NN}{\mbox{$\mathbb N$}}
\newcommand{\ZZ}{\mbox{$\mathbb Z$}}
\newcommand{\eopf}{\raisebox{0.8ex}{\framebox{}}}
\newcommand{\dist}{\hbox{\rm d}}
\renewcommand\a{\alpha}
\renewcommand\b{\beta}
\renewcommand\c{\gamma}
\renewcommand\d{\delta}
\newcommand\D{\Delta}
\newcommand{\directedchi}{\mbox{$\vec{\chi}$}}
\newcommand{\directedE}{\mbox{$\vec{E}$}}
\newcommand{\directedG}{\mbox{$\vec{G}$}}
\newcommand{\directedK}{\mbox{$\vec{K}$}}

\newenvironment{proof}%
{\noindent{\bf Proof.}\ }%
{\hfill\eopf\par\bigskip}%

%%%%%%%%%%%%%%%%%%%%%%%%%%%%%%%%%%%%%%%%%%%%%%%%%%%%%%%%

%\maketitle


\noindent{\bf  Last Name:} $.............................$
\noindent{\bf First Name:} $..............................$
\noindent{\bf Email:}          $..............................$

\noindent
\noindent{ CS 6505, Fall 2017, Homework 1, 9/1/17 Due 9/8/17 in class $~~$Page 1/10 }\\

\noindent
{\bf Problem 1, Analysis of Algorithm (10 points)}\\
Where $n$ is a power or 2 and $n > 8$, how many $x$'s does the function Mystery$(n)$ below print?\\
(a) Write and solve the recurrence exactly using substitution.\\
(b) State the solution in $O()$ notation. \\
$~~~~$ Mystery$(n)$\\
$~~~~$ if $n>8 $ then begin\\
$~~~~~~~~~~~~~~~~~~~~~~~$ print$("x")$\\
$~~~~~~~~~~~~~~~~~~~~~~~$ Mystery$\left(  \frac{n}{2}  \right)$\\
$~~~~~~~~~~~~~~~~~~~~~~~$ Mystery$\left(  \frac{n}{2}  \right)$\\
$~~~~~~~~~~~~~~~~~~~~~~~$ end\\

\noindent
{\bf Answer:}

\pagebreak
\noindent{\bf  Last Name:} $.............................$
\noindent{\bf First Name:} $..............................$
\noindent{\bf Email:}          $..............................$

\noindent
\noindent{ CS 6505, Fall 2017, Homework 1, 9/1/17 Due 9/8/17 in class $~~$Page 2/10 }\\

\noindent
{\bf Problem 2, Analysis of Algorithm (10 points)}\\
Where $n$ is a power or 2 and $n \geq 1$, how many $x$'s does the function Mystery$(n)$ below print?\\
(a) Write and solve the recurrence exactly using substitution.\\
(b) State the solution in $O()$ notation. \\
$~~~~$ Mystery$(n)$\\
$~~~~$ print$("xx")$\\
$~~~~$ if $n>1 $ then begin\\
$~~~~~~~~~~~~~~~~~~~~~~~$ Mystery$\left(  \frac{n}{2}  \right)$\\
$~~~~~~~~~~~~~~~~~~~~~~~$ Mystery$\left(  \frac{n}{2}  \right)$\\
$~~~~~~~~~~~~~~~~~~~~~~~$ Mystery$\left(  \frac{n}{2}  \right)$\\
$~~~~~~~~~~~~~~~~~~~~~~~$ end\\

\noindent
{\bf Answer:}

\pagebreak
\noindent{\bf  Last Name:} $.............................$
\noindent{\bf First Name:} $..............................$
\noindent{\bf Email:}          $..............................$

\noindent
\noindent{ CS 6505, Fall 2017, Homework 1, 9/1/17 Due 9/8/17 in class $~~$Page 3/10 }\\

\medskip\noindent
{\bf Problem 3, Analysis of Algorithm (10 points)}\\
Where $n$ is a power or 2 and $n \geq 1$, how many $x$'s does the function Mystery$(n)$ below print?\\
(a) Write and solve the recurrence exactly using substitution.\\
(b) State the solution in $O()$ notation. \\
$~~~~$ Mystery$(n)$\\
$~~~~$ for $i:=1$ to $n^2$ \\
$~~~~~~~~~~~~$ print$("xx")$\\
$~~~~$ if $n>1 $ then begin\\
$~~~~~~~~~~~~~~~~~~~~~~~$ Mystery$\left(  \frac{n}{2}  \right)$\\
$~~~~~~~~~~~~~~~~~~~~~~~$ Mystery$\left(  \frac{n}{2}  \right)$\\
$~~~~~~~~~~~~~~~~~~~~~~~$ Mystery$\left(  \frac{n}{2}  \right)$\\
$~~~~~~~~~~~~~~~~~~~~~~~$ end\\

\noindent
{\bf Answer:}

\pagebreak
\noindent{\bf  Last Name:} $.............................$
\noindent{\bf First Name:} $..............................$
\noindent{\bf Email:}          $..............................$

\noindent
\noindent{ CS 6505, Fall 2017, Homework 1, 9/1/17 Due 9/8/17 in class $~~$Page 4/10 }\\


\medskip\noindent
{\bf Problem 4, Analysis of Algorithm (10 points)}\\
Where $n$ is a positive integer,  how many $x$'s does the function Mystery$(n)$ below print?\\
(a) Write and solve the recurrence exactly using substitution.\\
(b) State the solution in $O()$ notation. \\
$~~~~$ Mystery$(n)$\\
$~~~~$ if $n=1$ then  print$("x")$\\
$~~~~$ if $n>1 $ then begin\\
$~~~~~~~~~~~~~~~~~~~~~~~$ Mystery$\left(  n-1  \right)$\\
$~~~~~~~~~~~~~~~~~~~~~~~$ Mystery$\left(  n-1  \right)$\\
$~~~~~~~~~~~~~~~~~~~~~~~$ end\\

\noindent
{\bf Answer:}

\pagebreak
\noindent{\bf  Last Name:} $.............................$
\noindent{\bf First Name:} $..............................$
\noindent{\bf Email:}          $..............................$

\noindent
\noindent{ CS 6505, Fall 2017, Homework 1, 9/1/17 Due 9/8/17 in class $~~$Page 5/10 }\\


\medskip\noindent
{\bf Problem 5, Min and Max with Fewer Comparisons (10 points)}\\
Let $a_1 \ldots a_n$ be an input array of $n$ unsorted distint integers, where $n$ is an even number.
Let $m_{\max}$ be the maximum value of the above integers, and 
let $m_{\min}$ be the minimum value of the above integers.
Give an  algorithm that findsboth $m_{\max}$ and $m_{\min}$ using at most $\frac{3n}{2}\! - \! 2$ comparisons.\\
Argue correctness and running time of your solution.\\

\noindent
{\bf Answer:}

\pagebreak

\noindent{\bf  Last Name:} $.............................$
\noindent{\bf First Name:} $..............................$
\noindent{\bf Email:}          $..............................$\

\noindent
\noindent{ CS 6505, Fall 2017, Homework 1, 9/1/17 Due 9/8/17 in class $~~$Page 6/10 }\\

\medskip\noindent
{\bf Problem 6, Sorting Faster than $O(n\log n)$ in Special Case (10 points)}\\
Let $a_1 \ldots a_n$ be an input array of $n$ unsorted and not necessarily distint integers. 
Let $m_{\max}$ be the maximum value of the above integers, and 
let $m_{\min}$ be the minimum value of the above integers.
Let $M=m_{\max}-m_{\min}$.
Give an $O(n+M)$ comparison algorithm that sorts the input array.\\
Argue correctness and running time of your solution.\\

\noindent
{\bf Answer:}

\pagebreak
\noindent{\bf  Last Name:} $.............................$
\noindent{\bf First Name:} $..............................$
\noindent{\bf Email:}          $..............................$

\noindent
\noindent{ CS 6505, Fall 2017, Homework 1, 1/9/17 Due 9/8/17 in class $~~$Page 7/10 }\\


\medskip\noindent
{\bf Problem 7, Use of  Pointers (10 points)}\\
Let $a_1 \ldots a_n$ be $n$ sorted distint integers, 
and let $\tau$ be an additional given integer.
Give an $O(n)$ comparison algorithm that decides if there exist distinct indices $i$ and $j$,
ie $1\leq i < j \leq n$, such that $a_i+a_j=\tau$.\\
Argue correctness and running time of your solution.\\

\noindent
{\bf Answer:}


\pagebreak
\noindent{\bf  Last Name:} $.............................$
\noindent{\bf First Name:} $..............................$
\noindent{\bf Email:}          $..............................$

\noindent{ CS 6505, Fall 2017, Homework 1, 9/1/17 Due 9/8/17 in class $~~$Page 8/10 }\\


\noindent
{\bf Problem 8, Searching a Tree (10 points)}\\
You are given a complete binary tree on $n$ nodes, 
 where each node has a distinct value $w_i$, $1\leq i \leq n$. \\
The input representation is as follows:\\
$~~~~$(1) Index 1 is the root of the tree.\\
$~~~~$(2) For $1\leq i \leq \frac{n-1}{2}$, the left child of $i$ is $2i$ and the right child of $i$ is $2i+1$.\\
$~~~~$(3) For $2\leq i \leq n$, the parent of $i$ is $\lfloor \frac{n}{2} \rfloor$.\\
Say that $k$ is a {\em local minimum}  if and only if:\\
$~~~~$(1) If $k=1$, then $w_1$ is smaller than both its children.\\
$~~~~$(2) If $k\geq \frac{n-1}{2}$, then $w_k$ is smaller than its parent.\\
$~~~~$(3) If $2 \leq k \leq \frac{n-1}{2}$, then $w_k$ is smaller than both its children, and
$w_k$ is also smaller than its parent.\\
Give an $O(\log n)$ comparison algorithm that finds a local minimum of the binary tree.\\
Justify correctness and running time.\\

\noindent
{\bf Answer:}


\pagebreak
\noindent{\bf  Last Name:} $.............................$
\noindent{\bf First Name:} $..............................$
\noindent{\bf Email:}          $..............................$

\noindent{ CS 6505, Fall 2017, Homework 1, 9/1/17 Due 9/8/17 in class $~~$Page 9/10 }\\


\noindent
{\bf Problem 9,  Sorting in Linear Time in Special Case (10 points)}\\
Suppose that $n$ is a perfect square and let $N=n+\sqrt{n}$.\\
You are given an array of $a_1 \ldots a_N$ distinct integers, where the first $n$ integers are sorted
$a_1 < a_2 < \ldots < a_n$, but the last $\sqrt{n}$ integers are not sorted. 
Give an $O(N)$ comparison algorithm that sorts the entire input array $a_1 \ldots a_N$.\\

\noindent{\bf Answer:}

\pagebreak
\noindent{\bf  Last Name:} $.............................$
\noindent{\bf First Name:} $..............................$
\noindent{\bf Email:}          $..............................$

\noindent{ CS 6505, Fall 2017, Homework 1, 9/1/17 Due 9/8/17 in class $~~$Page 10/10 }\\


\noindent
{\bf Problem 10, Comparing Algorithmic Performance (10 points)}\\
Suppose you are choosing between the following three algorithms:\\
$\bullet$ Algorithm A solves problems by dividing them into five subproblems of half the size, 
recursively solving each subproblem, and then combining solutions in linear time.\\
$\bullet$  Algorithm B solves problems of size $n$ by recursively solving two subproblems of size $(n\! - \! 1)$
and then combining the solutions in constant time.\\
$\bullet$ Algorithm C solves problems of size $n$ by dividing them into nine subproblems of size $n/3$,
recursively solving each subproblem, and then combining the solutions in $O(n^2)$ time.\\
Solve each recurrence by substitution, and give the running times of each of these algorithms 
in $O()$ notation. Which one would you choose as the asymptotically fastest? \\

\noindent{\bf Answer:}








\end{document}

